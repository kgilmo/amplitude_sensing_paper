%% LyX 2.2.1 created this file.  For more info, see http://www.lyx.org/.
%% Do not edit unless you really know what you are doing.
\documentclass[english]{article}
\usepackage[T1]{fontenc}
\usepackage[latin9]{inputenc}
\usepackage{amssymb}
\usepackage{babel}
\begin{document}

\part*{Incoherent sensing limits}

Following earlier discussions, the probability of measuring $\left|\uparrow\right\rangle $
at the end of the Ramsey sequence is 
\begin{equation}
\left\langle P_{\uparrow}\right\rangle =\frac{1}{2}\left[1-e^{-\Gamma\tau}J_{0}\left(\theta_{max}\right)\right]\:,\label{eq:P_up formula}
\end{equation}
where $\left\langle \:\right\rangle $ denotes an average over many
experimental trials and therefore over the random phase between the
1D optical lattice and the classically driven COM motion, and
\begin{equation}
\theta_{max}=DWF\cdot\left(U/\hbar\right)\cdot\delta k\,Z_{c}\cdot\tau\:.\label{eq:theta_max formula}
\end{equation}
Defining $F\left(\theta_{max}^{2}\right)\equiv\left(1-J_{0}\left(\theta_{max}\right)\right)/2$
and denoting $\left\langle P_{\uparrow}\right\rangle _{bck}=\left[1-e^{-\Gamma\tau}\right]/2$
as the probability of measuring $\left|\uparrow\right\rangle $ at
the end of the sequence in the absence of a classically driven motion,
$\theta_{max}^{2}$ can be determined from a measurement of the difference
$\left\langle P_{\uparrow}\right\rangle -\left\langle P_{\uparrow}\right\rangle _{bck}$
through
\begin{equation}
F\left(\theta_{max}^{2}\right)=e^{\Gamma\tau}\left(\left\langle P_{\uparrow}\right\rangle -\left\langle P_{\uparrow}\right\rangle _{bck}\right)\:.\label{eq:P_up difference}
\end{equation}
The standard deviation $\delta\theta_{max}^{2}$ in estimating $\theta_{max}^{2}$
is determined from the standard deviation $\delta\left(\left\langle P_{\uparrow}\right\rangle -\left\langle P_{\uparrow}\right\rangle _{bck}\right)$
of the $\left\langle P_{\uparrow}\right\rangle -\left\langle P_{\uparrow}\right\rangle _{bck}$
difference measurements through
\begin{equation}
\delta\theta_{max}^{2}=\frac{e^{\Gamma\tau}\delta\left(\left\langle P_{\uparrow}\right\rangle -\left\langle P_{\uparrow}\right\rangle _{bck}\right)}{\frac{\mathrm{d}F\left(\theta_{max}^{2}\right)}{d\theta_{max}^{2}}}\:.\label{eq:delta theta^2}
\end{equation}
The signal-to-noise ratio of a measurement of $\theta_{max}^{2}$
(and therefore $Z_{c}^{2}$) is then $\theta_{max}^{2}/\delta\theta_{max}^{2}$.
In general this signal-to-noise ratio depends on $\theta_{max}^{2}$
and the experimental parameters $U\cdot\tau$ and $\Gamma\cdot\tau$$ $.

We use Eq. (\ref{eq:delta theta^2}) to theoretically estimate $\theta_{max}^{2}/\delta\theta_{max}^{2}$
and the amplitude sensing limits. We assume the only sources of noise
are projection noise in the measurement of the spin state and fluctuations
in $P_{\uparrow}$ due to the random variation in the relative phase
of the 1D optical lattice and the driven COM motion. In this case
$\delta\left(\left\langle P_{\uparrow}\right\rangle -\left\langle P_{\uparrow}\right\rangle _{bck}\right)=\sqrt{\delta\left\langle P_{\uparrow}\right\rangle ^{2}+\delta\left\langle P_{\uparrow}\right\rangle _{bck}^{2}}$
where the relevant variances are
\begin{equation}
\delta\left\langle P_{\uparrow}\right\rangle _{bck}^{2}=\frac{1}{N}\left\langle P_{\uparrow}\right\rangle _{bck}\left(1-\left\langle P_{\uparrow}\right\rangle _{bck}\right)=\frac{1}{4N}\left(1-e^{-2\Gamma\tau}\right)\label{eq:Pbck noise}
\end{equation}
and
\begin{equation}
\delta\left\langle P_{\uparrow}\right\rangle ^{2}=\sigma_{\phi}^{2}+\frac{1}{N}\left\langle P_{\uparrow}\right\rangle \left(1-\left\langle P_{\uparrow}\right\rangle \right)\:.\label{eq:Pup noise}
\end{equation}
Here $N$ is the number of spins. Equation (\ref{eq:Pbck noise})
and the second term in Eq. (\ref{eq:Pup noise}) are projection noise.
The variance 
\begin{equation}
\sigma_{\phi}^{2}=\left\langle P_{\uparrow}^{2}-\left\langle P_{\uparrow}\right\rangle ^{2}\right\rangle =\frac{e^{-2\Gamma\tau}}{8}\left(1+J_{0}\left(2\theta_{max}\right)-2J_{0}\left(\theta_{max}\right)^{2}\right)\label{eq:phase fluctuation variance}
\end{equation}
is due to the random variation in the relative phase of the 1D optical
lattice and the driven COM motion. For our set-up, $DWF=0.86$ and
$\delta k=2\pi/\left(900\:\mathrm{nm}\right)$ are fixed, and the
decoherence $\Gamma$ is a function of $U$, $\Gamma=\xi\left(U/\hbar\right)$
where $\xi=1.156\times10^{-3}$. For a given $Z_{c}$ we use Eqs.
(\ref{eq:theta_max formula}) and (\ref{eq:delta theta^2})-(\ref{eq:phase fluctuation variance})
to find the optimum $\theta_{max}^{2}/\delta\theta_{max}^{2}$ as
a function of $\left(U\tau\right)/\hbar$. This optimum value is the
theoretical curve plotted in Figure (amplitude sensing limits) of
the main text.

One can show that $\theta_{max}^{2}/\delta\theta_{max}^{2}$ is optimized
for relatively small values of $\theta_{max}^{2}$ where $F\left(\theta_{max}^{2}\right)\approx\theta_{max}^{2}/8$
is a good approximation. This leads to some simplifications for Eqs.
(\ref{eq:P_up difference}) and (\ref{eq:delta theta^2}),
\begin{equation}
\theta_{max}^{2}\approx8e^{\Gamma\tau}\left(\left\langle P_{\uparrow}\right\rangle -\left\langle P_{\uparrow}\right\rangle _{bck}\right)\label{eq:theta_max^sq}
\end{equation}
and
\begin{equation}
\delta\theta_{max}^{2}\approx8e^{\Gamma\tau}\delta\left(\left\langle P_{\uparrow}\right\rangle -\left\langle P_{\uparrow}\right\rangle _{bck}\right)\:,\label{eq:delta theta_max^2}
\end{equation}
and to the following estimate for the signal-to-noise ratio of a single
measurement,
\begin{equation}
\frac{\theta_{max}^{2}}{\delta\theta_{max}^{2}}\approx\frac{\left\langle P_{\uparrow}\right\rangle -\left\langle P_{\uparrow}\right\rangle _{bck}}{\delta\left(\left\langle P_{\uparrow}\right\rangle -\left\langle P_{\uparrow}\right\rangle _{bck}\right)}\:.\label{eq:S/N limits}
\end{equation}
Figure (amplitude sensing limits) of the main text uses Eq. (\ref{eq:S/N limits}),
along with repeated measurements of $\left\langle P_{\uparrow}\right\rangle -\left\langle P_{\uparrow}\right\rangle _{bck}$,
to experimentally determine the signal-to-noise ratio as a function
of the amplitude $Z_{c}$ of the COM motion.

Finally we use Eqs. (\ref{eq:theta_max formula}) and (\ref{eq:delta theta^2})-(\ref{eq:phase fluctuation variance})
to calculate the sensing limits for very small $Z_{c}$. For small
$Z_{c}$ the variance $\sigma_{\phi}^{2}$ can be neglected compared
to projection noise and $\delta\left\langle P_{\uparrow}\right\rangle ^{2}\approx\delta\left\langle P_{\uparrow}\right\rangle _{bck}^{2}$.
In this case we obtain the following expression for the signal-to-noise
ratio,
\begin{equation}
\frac{\theta_{max}^{2}}{\delta\theta_{max}^{2}}=\frac{\sqrt{N}}{4\sqrt{2}}\frac{DWF^{2}\cdot\left(\delta k\,Z_{c}\right)^{2}\left(U\tau/\hbar\right)^{2}}{\sqrt{e^{2\xi U\tau/\hbar}-1}}\:.\label{eq:limiting sensitivity}
\end{equation}
Equation (\ref{eq:limiting sensitivity}) is maximized for $\xi U\tau\approx1.9603$.
With $DWF=0.86$, $\delta k=2\pi/\left(900\:\mathrm{nm}\right)$,
$\xi=1.156\times10^{-3}$, and $N=100$,
\begin{equation}
\left.\frac{\theta_{max}^{2}}{\delta\theta_{max}^{2}}\right|_{optimum}=\left[\frac{Z_{c}}{0.196\:\mathrm{nm}}\right]^{2}\:.\label{eq:26.1 Z_c^2}
\end{equation}
For our set-up and available ODF power, $\xi U\tau/\hbar\approx1.9603$
is realized for $\tau\approx20\:\mathrm{ms}$. A measurement of the
signal and a measurement of the background requires $\sim60\:\mathrm{ms}$
for $16$ independent measurements of $\left\langle P_{\uparrow}\right\rangle -\left\langle P_{\uparrow}\right\rangle _{bck}$
in 1 s. The limiting sensitivity is approximately $\left(100\:\mathrm{pm}\right)^{2}$
in a 1 s measurement time, or $\left(100\:\mathrm{pm}\right)^{2}/\sqrt{\mathrm{Hz}}$.
We note that the limiting sensitivity is determined by the ratio $\xi=\Gamma/\left(U/\hbar\right)$.
In particular, the optimum value for Eq. (\ref{eq:limiting sensitivity})
scales as $1/\xi^{2}$.

\part*{Coherent sensing limits}

With appropriate care the 1D optical lattice can be stable for long
periods of time with respect to the ion trapping electrodes {[}Hume,
PRL{]}, enabling phase coherent sensing of a COM motion $Z_{c}\cos(\omega t)$.
In this case the same spin precession $\theta_{max}=DWF\cdot\left(U/\hbar\right)\cdot\delta k\,Z_{c}\cdot\tau$
occurs for each experimental trial, which can be detected to first
order in $\theta_{max}$ (or $Z_{c}$) in a Ramsey sequence with a
$\pi/2$ phase shift between the two $\pi$-pulses. Assuming $\sin\left(\theta_{max}\right)\approx\theta_{max}$,
appropriate for small amplitudes $Z_{c}$, the equivalent coherent
sensing expressions for Eqs. (\ref{eq:theta_max^sq}) and (\ref{eq:delta theta_max^2})
are
\begin{equation}
\theta_{max}=2e^{\Gamma\tau}\left(\left\langle P_{\uparrow}\right\rangle -\left\langle P_{\uparrow}\right\rangle _{bck}\right)\label{eq:theta_max coherent}
\end{equation}
and
\begin{equation}
\delta\theta_{max}=2e^{\Gamma\tau}\delta\left(\left\langle P_{\uparrow}\right\rangle -\left\langle P_{\uparrow}\right\rangle _{bck}\right)\:.\label{eq:delta theta_max coherent}
\end{equation}
For a Ramsey experiment with a $\pi/2$ phase shift, $\left\langle P_{\uparrow}\right\rangle _{bck}=1/2$.
If projection noise is the only source of noise, then for small $Z_{c}$,
$\delta\left\langle P_{\uparrow}\right\rangle ^{2}\approx\delta\left\langle P_{\uparrow}\right\rangle _{bck}^{2}=\frac{1}{N}\cdot\frac{1}{2}\cdot\frac{1}{2}$
and $\delta\left(\left\langle P_{\uparrow}\right\rangle -\left\langle P_{\uparrow}\right\rangle _{bck}\right)\approx\frac{1}{\sqrt{2N}}$.
The limiting signal-to-noise ratio $\theta_{max}/\delta\theta_{max}$
of a $\left(\left\langle P_{\uparrow}\right\rangle -\left\langle P_{\uparrow}\right\rangle _{bck}\right)$
measurement is
\begin{equation}
\frac{\theta_{max}}{\delta\theta_{max}}=DWF\cdot\left(\delta k\,Z_{c}\right)\cdot\sqrt{\frac{N}{2}}\cdot\frac{\left(U\tau\right)}{\hbar}e^{-\xi U\tau/\hbar}\:.\label{eq:S/N coherent}
\end{equation}
Equation (\ref{eq:S/N coherent}) is maximized for $\xi U\tau/\hbar=1$.
With $DWF=0.86$, $\delta k=2\pi/\left(900\:\mathrm{nm}\right)$,
$\xi=1.156\times10^{-3}$, and $N=100$,
\begin{equation}
\left.\frac{\theta_{max}}{\delta\theta_{max}}\right|_{optimum}=\frac{Z_{c}}{0.074\:\mathrm{nm}}\:.\label{eq:coherent optimum}
\end{equation}
With $16$ independent measurements of $\left\langle P_{\uparrow}\right\rangle -\left\langle P_{\uparrow}\right\rangle _{bck}$
in 1 s, this corresponds to a limiting sensitivity of $\sim\left(20\:\mathrm{pm}\right)/\sqrt{\mathrm{Hz}}$.
In particular the optimum value for the signal-to-ratio of Eq. (\ref{eq:S/N coherent})
scales as $1/\xi$. Further, by employing spin-squeezed states that
have been demonstrated in this system {[}Bohnet, Science 2016{]},
$\theta_{max}/\delta\theta_{max}$ can be improved by another vactor
of 2.

Employing this technique to sense motion resonance with the COM mode
can lead to the detection of very weak forces and electric fields.
The detection of a 20 pm amplitude resulting from a 5 ms coherent
drive on the 1.57 MHz COM mode (corresponds to a $Q\sim10^{4}$) is
sensitive to a force/ion of $10^{-3}\:\mathrm{yN}$ corresponding
to an electric field of $7\:\mathrm{nV/m}$.
\end{document}
